\begin{thesisresumo}
Este trabalho apresenta o desenvolvimento de um site institucional responsivo para portfólio fotográfico, com o objetivo de oferecer ao fotógrafo profissional Geovany uma plataforma leve e de fácil manutenção para exibir seus trabalhos e captar clientes. A pesquisa se insere no contexto do marketing digital para profissionais criativos, considerando a importância de uma presença online otimizada. O projeto foi realizado no contexto da disciplina de Análise e Projeto de Sistemas da Universidade Positivo, adota-se uma abordagem pragmática, baseada em metodologias ágeis (Scrum/Kanban) e boas práticas de UX/UI, aliada ao uso de ferramentas como Figma, Trello e GitHub. Foram realizados workshops de levantamento de requisitos, prototipação de interfaces e implementação do front-end em HTML5, CSS3 e JavaScript, com técnicas de otimização de desempenho. Contribui para o fortalecimento da visibilidade online de fotógrafos e para o aprendizado prático de gestão de projetos e desenvolvimento web.

\ \\

% use de três a cinco palavras-chave

\textbf{Palavras-chave}: Desenvolvimento Web, portfólio fotográfico, responsividade, metodologias ágeis, UX/UI

\end{thesisresumo}
