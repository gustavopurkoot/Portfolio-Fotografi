\chapter{Conceito do projeto do portfólio}
\label{chap:fundteor}
%--------- NEW SECTION ----------------------
Com o crescimento da presença digital, profissionais autônomos como fotógrafos passaram a depender cada vez mais de plataformas online para divulgar seus serviços e alcançar novos clientes. Um site institucional cumpre exatamente esse papel, funcionando como uma vitrine digital que reúne informações sobre o profissional, formas de contato e, principalmente, seu portfólio de trabalhos. Nesse contexto, este projeto tem como foco o desenvolvimento de um site responsivo para o fotógrafo Geovany, permitindo que seu trabalho seja apresentado de forma organizada, elegante e acessível em diferentes dispositivos. O site foi planejado com base em boas práticas de usabilidade (UX), acessibilidade e design responsivo, assegurando uma experiência agradável ao usuário. Além disso, a estrutura do projeto seguiu o modelo de desenvolvimento em cascata (Waterfall), com etapas bem definidas como levantamento de requisitos, prototipação, codificação, testes e entrega. Foram utilizadas ferramentas como Figma para o design das interfaces, Trello para organização das tarefas e GitHub para versionamento do código. A implementação utilizou tecnologias web como HTML5, CSS3 e JavaScript, priorizando leveza e desempenho. Dessa forma, o projeto alia conceitos de desenvolvimento web com necessidades reais de um profissional da fotografia, resultando em uma solução prática, eficiente e alinhada ao mercado.
% isso é igual <=  === <> #{ #( www
% <| |>
% ===

Lista dos documentos
\begin{enumerate}
   \item diagrama de classe
   \item diagrama de casos de uso
   \item diagrama de sequência
\end{enumerate}

O desenvolvimento deste projeto consiste na criação de um site institucional responsivo para um fotógrafo profissional. O objetivo é disponibilizar uma plataforma online que permita divulgar seu portfólio, facilitar o contato com clientes e fortalecer sua presença digital por meio de um layout moderno, leve e funcional.

Neste capítulo serão abordados os requisitos do cliente, os requisistos técnicos, a criação do portfólio e a pesquisa por similares. 



%conferir se precisa de requisitos do cliente
\section{Requisitos do cliente}
 O cliente definiu certos requisitos quanto ao projeto do portfólio fotográfico, que são:
 \begin{itemize}
    \item RNF01 - O sistema deve ser responsivo, funcionando bem em celulares, tablets e desktops;
    \item RNF01 - O sistema deve ser responsivo, funcionando bem em celulares, tablets e desktops;
    \item RNF02 - As imagens devem ser otimizadas para carregamento rápido;
    \item RNF03 - O - site deve ter uma interface intuitiva e acessível para qualquer usuário;
    \item RNF04 - O backend deve ser desenvolvido com boas práticas de segurança (validação de dados, CORS, etc.);
    \item RNF05 - O sistema deve ser capaz de escalar facilmente (ex: uso de serviços em nuvem);
    \item RNF06 - O código deve estar organizado e documentado para facilitar manutenção e evolução;
    \item RNF07 - O sistema deve ter tempo de resposta rápido (até 2 segundos para carregamento de páginas);
    \item RNF08 - O deploy do sistema deve ser feito em um serviço confiável (ex: Vercel, Render, Netlify, Railway).
    














































































 \end{itemize}

\section{Requisitos funcionais}
 \begin{itemize}
   \item RF01 - Exibir galeria de fotos organizada por categorias (ex: Casamento, Natureza, Retratos);
   \item RF02 - Permitir ao usuário filtrar fotos por categoria;
   \item RF03 - Possibilitar o agendamento de sessões fotográficas com data, horário e tipo de ensaio;
   \item RF04 - Armazenar e listar agendamentos feitos pelos clientes;
   \item RF05 - Disponibilizar uma lista de posts no blog, com título, resumo e imagem;
   \item RF06 - Exibir detalhes de cada post do blog ao clicar;
   \item RF07 - Permitir que o fotógrafo publique novos posts (via painel ou API);
   \item RF08 - Enviar confirmação de agendamento por e-mail;
   \item RF09 - Painel administrativo para o fotógrafo gerenciar fotos, posts e agendamentos.




    \end{itemize}

%  \section{Missão}
%  \lipsum
%  %desenvolver mais
%  Além disso, o Walker deve realizar um desafio, que consiste em navegar de forma autônoma, se localizar por meio de tags e encontrar um determinado objeto.



%  \section{Pesquisa por similares}


% %----------------------------------------------------------

% %--------- NEW SECTION ----------------------


% %---------------picture------------------------------------
% % \begin{figure}
% %     \centering
% %     \subfigure[Figure A]{\label{fig:a}\includegraphics[width=60mm]{./lq}}
% %     \subfigure[Figure B]{\label{fig:b}\includegraphics[width=60mm]{./lq}}
% %     \subfigure[Figure C]{\label{fig:c}\includegraphics[width=\textwidth]{./lq}}
% %     \caption{Three simple graphs}
% %     \label{fig:three graphs}
% % \end{figure}
% %----------------------------------------------------------

% % \begin{figure}
% %     \centering
% %     \begin{subfigure}[b]{0.3\textwidth}
% %         \centering
% %         \includegraphics[width=\textwidth]{./lq}
% %         \caption{$y=x$}
% %         \label{fig:y equals x}
% %     \end{subfigure}
% %     \hfill
% %     \begin{subfigure}[b]{0.3\textwidth}
% %         \centering
% %         \includegraphics[width=\textwidth]{./lq}
% %         \caption{$y=3sinx$}
% %         \label{fig:three sin x}
% %     \end{subfigure}
% %     \hfill
% %     \begin{subfigure}[b]{0.3\textwidth}
% %         \centering
% %         \includegraphics[width=\textwidth]{./lq}
% %         \caption{$y=5/x$}
% %         \label{fig:five over x}
% %     \end{subfigure}
% %        \caption{Three simple graphs}
% %        \label{fig:three graphs}
% % \end{figure}


% % %--------- NEW SECTION ----------------------
% % \section{Assunto 2}
% % \label{sec:ass2}
% % flkjasdlkfjasdlkfjs

% % \begin{table}[h]
% %     \begin{subtable}[h]{0.45\textwidth}
% %         \centering
% %         \begin{tabular}{l | l | l}
% %         Day & Max Temp & Min Temp \\
% %         \hline \hline
% %         Mon & 20 & 13\\
% %         Tue & 22 & 14\\
% %         Wed & 23 & 12\\
% %         Thurs & 25 & 13\\
% %         Fri & 18 & 7\\
% %         Sat & 15 & 13\\
% %         Sun & 20 & 13
% %        \end{tabular}
% %        \caption{First Week}
% %        \label{tab:week1}
% %     \end{subtable}
% %     \hfill
% %     \begin{subtable}[h]{0.45\textwidth}
% %         \centering
% %         \begin{tabular}{l | l | l}
% %         Day & Max Temp & Min Temp \\
% %         \hline \hline
% %         Mon & 17 & 11\\
% %         Tue & 16 & 10\\
% %         Wed & 14 & 8\\
% %         Thurs & 12 & 5\\
% %         Fri & 15 & 7\\
% %         Sat & 16 & 12\\
% %         Sun & 15 & 9
% %         \end{tabular}
% %         \caption{Second Week}
% %         \label{tab:week2}
% %      \end{subtable}
% %      \caption{Max and min temps recorded in the first two weeks of July}
% %      \label{tab:temps}
% % \end{table}
