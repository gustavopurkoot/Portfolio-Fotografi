\chapter{Conclusão}
\label{chap:conc}

O desenvolvimento do site institucional para o fotógrafo Geovany demonstrou a viabilidade de criar uma plataforma leve, responsiva e de fácil manutenção, alinhada às necessidades de divulgação de portfólio e captação de clientes. Ao longo das etapas do modelo Waterfall — levantamento de requisitos, prototipação, implementação, testes e entrega — foi possível mapear com clareza as demandas do cliente e traduzi-las em funcionalidades concretas: galeria filtrável por categorias, sistema de agendamento de ensaios, formulário de contato integrado ao e-mail e WhatsApp, além de integrações com Instagram e meios de pagamento.

A adoção de boas práticas de UX/UI e otimização de performance (compressão de imagens e lazy loading) garantiu um tempo médio de carregamento abaixo de 3 segundos, conforme previsto nos requisitos técnicos. A prototipação no Figma e o uso de Kanban no Trello facilitaram a comunicação entre a equipe e a obtenção de feedbacks rápidos do cliente, resultando em um produto final que atende plenamente ao escopo definido.



\section{Considerações finais}
\label{sec:consid}

A experiência proporcionada por este projeto reforça a importância de um processo de desenvolvimento estruturado e centrado no usuário. Do ponto de vista acadêmico, o trabalho permitiu aplicar conceitos de gestão de projetos, modelagem de requisitos e desenvolvimento web em um caso real, promovendo o aprendizado colaborativo e prático.  

Tecnicamente, a entrega de um site responsivo com funcionalidades completas — incluindo agendamento online e integração com serviços externos — agrega valor ao fotógrafo, ampliando sua visibilidade digital e potencial de negócios. Socialmente, contribui para a profissionalização de pequenos empreendedores criativos, que dependem cada vez mais de soluções digitais acessíveis.

Em termos de perspectivas futuras, recomenda-se explorar a inclusão de um painel administrativo mais robusto, análise de métricas de uso (via Google Analytics) e evolução para uma versão multilíngue. Essas melhorias podem ampliar o alcance do site e oferecer insights valiosos para o cliente sobre o comportamento dos visitantes.










