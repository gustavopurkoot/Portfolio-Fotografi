\chapter{Introdução}
\label{chap:intro}

% Este pode ser um parágrafo citado por alguém \cite{Barabasi2003-1} e \cite{barabasi2003linked}.
% Para ajustar veja o comentário do capítulo \ref{chap:fundteor}.

% As orientações do robô \cite{aperea-1}.

% fakdfjlsdjfldsjfldsj
% dfkhfdskfhkdjh


% Segundo \citeonline{barabasi2003linked}, ...

% 
% \loremipsum dolor sit amet, consectetur adipiscing elit. Sed do eiusmod tempor incididunt ut labore et dolore magna aliqua. Ut enim ad minim veniam, quis nostrud exercitation ullamco laboris nisi ut aliquip ex ea commodo consequat. Duis aute irure dolor in reprehenderit in voluptate velit esse cillum dolore eu fugiat nulla pariatur. Excepteur sint occaecat cupidatat non proident, sunt in culpa qui officia deserunt mollit anim id est laborum.
%--------- NEW SECTION ----------------------
\section{Objetivos}
\label{sec:obj}
Este projeto tem como objetivo principal desenvolver um site institucional responsivo para o fotógrafo profissional Geovany, com foco na divulgação de seu portfólio, facilitação do contato com clientes e modernização de sua presença digital. O site deve ser leve, visualmente atrativo, de fácil navegação e manutenção, atendendo tanto usuários em computadores quanto em dispositivos móveis.
\label{sec:obj}

\subsection{Objetivos Específicos}
\label{ssec:objesp}
Os objetivos específicos deste projeto são:
\begin{itemize}
      \item Desenvolver habilidades de gestão de projetos.
      \item Levantar os requisitos funcionais e não funcionais por meio de reuniões e workshops com o cliente;
      \item Desenvolver o front-end utilizando HTML5, CSS3 e JavaScript;
      \item Criar uma galeria de fotos com filtro por categorias;
      \item Implementar um formulário de contato com integração por e-mail e WhatsApp;
      \item Garantir a responsividade do site para diferentes dispositivos;
      \item Realizar testes de usabilidade e performance;
  \end{itemize}

\subsubsection*{Objetivos específicos principais}
\label{sssec:obj-principais}
Os objetivos específicos principais deste projeto são:

\begin{itemize}
    \item Entregar um site responsivo com layout aprovado pelo cliente;
    \item Garantir carregamento rápido (menos de 3 segundos por página);
    \item Atender todos os requisitos definidos nas fases de levantamento e validação.
\end{itemize}


%--------- NEW SECTION ----------------------
\section{Justificativa}
\label{sec:justi}

Este projeto de desenvolvimento de um site institucional para fotógrafo profissional foi concebido para atender demandas reais do mercado, ao mesmo tempo em que consolida os conhecimentos acadêmicos nas disciplinas de Análise e Projeto de Sistemas e Gestão de Projetos de Software. A seguir apresentamos os principais impactos e justificativas:

Impacto Científico e Tecnológico:
- Aplicação prática dos conceitos de engenharia de software no levantamento de requisitos e modelagem UML
- Desenvolvimento de uma arquitetura web otimizada utilizando tecnologias padrão (HTML5, CSS3, JavaScript)
- Implementação de técnicas de otimização de desempenho e acessibilidade

Impacto Econômico:
- Solução de baixo custo para pequenos empreendedores da área criativa
- Potencial de aumento na captação de clientes para o profissional
- Redução de custos com hospedagem e manutenção

Impacto Social:
- Democratização do acesso ao trabalho artístico profissional
- Interface acessível para diferentes perfis de usuários
- Integração com redes sociais ampliando o alcance

Impacto Ambiental:
- Otimização de recursos computacionais
- Baixo consumo energético pela simplicidade da arquitetura
- Redução da pegada digital

Metodologia:
O projeto foi desenvolvido com base em:
1. Processos de gestão de projetos ágeis
2. Técnicas de design thinking
3. Boas práticas de desenvolvimento web
4. Controle de qualidade através de testes sistemáticos

Todos os argumentos apresentados foram validados empiricamente através de:
- Testes de usabilidade
- Feedback do cliente
- Análise comparativa com soluções similares

A solução desenvolvida evita promessas irreais, mantendo-se dentro do escopo tecnológico definido e das capacidades da equipe, enquanto oferece um produto completo e profissional para o cliente final.

\newpage


%--------- NEW SECTION ----------------------
\section{Organização do documento}
\label{section:organizacao}

Este documento apresenta $5$ capítulos e está estruturado da seguinte forma:

\begin{itemize}

  \item \textbf{Capítulo \ref{chap:intro} - Introdução}: Contextualiza o âmbito, no qual a pesquisa proposta está inserida. Apresenta, portanto, a definição do problema, objetivos e justificativas da pesquisa e como este \thetypeworkthree está estruturado;
  \item \textbf{Capítulo \ref{chap:fundteor} - Fundamentação Teórica}:  Apresenta os principais conceitos relacionados ao desenvolvimento web, design responsivo, usabilidade, metodologias ágeis e ferramentas utilizadas no projeto;
  \item \textbf{Capítulo \ref{chap:metod} - Materiais e Métodos}: Descreve a metodologia adotada (modelo Waterfall), os processos de levantamento de requisitos, prototipação, implementação, além das ferramentas utilizadas durante o desenvolvimento;
  \item \textbf{Capítulo \ref{chap:result} - Resultados}: Apresenta o produto final desenvolvido, as funcionalidades implementadas, os testes realizados, o feedback do cliente e a validação do projeto;
  \item \textbf{Capítulo \ref{chap:conc} - Conclusão}: Apresenta as conclusóes, contribuições e algumas sugestões de atividades de pesquisa a serem desenvolvidas no futuro.

\end{itemize}
